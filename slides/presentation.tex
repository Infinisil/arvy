\documentclass[12pt,hyperref={pdfpagelabels=false},usenames,dvipsnames]{beamer}
\let\Tiny=\tiny

\usepackage{xspace}
\usepackage{xmpmulti}
\usepackage{array}
\usepackage[absolute,overlay]{textpos}
\usepackage{forloop}
\usepackage{pgfplots}
\usepackage{subcaption}
\usepackage{tikz}
\usepackage{tabularx}
\usepackage{multirow}
\usepackage{numprint}
\usepackage{array}
\usetikzlibrary{pgfplots.groupplots}
\usetikzlibrary{calc}
\usetikzlibrary{positioning}
\usetikzlibrary{arrows.meta}
\usetikzlibrary{decorations.markings}
\usetikzlibrary{matrix}

\usepackage{biblatex}
\addbibresource{presentation.bib}

%\def\tabularxcolumn#1{m{#1}}
\renewcommand{\tabularxcolumn}[1]{>{\small}m{#1}}

\setbeamerfont{frametitle}{series=\bfseries}
\setbeamerfont{title}{series=\bfseries,size=\huge}
\setbeamerfont{author}{series=\bfseries}

\setbeamertemplate{navigation symbols}{}

\title{Arvy Heuristics for \\
Distributed Mutual Exclusion}
\author{Silvan Mosberger}
\institute{ETH Zurich -- Distributed Computing Group -- www.disco.ethz.ch}

\begin{document}


\newcommand{\evalTime}{Average request time}
\newcommand{\evalEdges}{Average tree edge distance}

\tikzset{
  rd/.style = { red!70!black },
  bl/.style = { blue!70!black },
  gr/.style = { green!50!black },
  cand/.style = { dashed, black!70!white, ->, >={Stealth[scale=1]} },
  arvy-expl/.style =
    { v/.style = { circle, draw } % normal graph vertices
    , r/.style = { v, gr } % root nodes
    , q/.style = { v, rd } % currently making request
    , e/.style = { draw=black, -> }
    , re/.style = { postaction={decorate}, >={Stealth[scale=1.5]}, densely dotted, bl }
    , weight/.style = { dotted, black!30!white }
    , every node/.style = { inner sep=0pt, minimum size=18pt },
    , >={Stealth[scale=2]}
    , scale = 1.0
    , auto,
    decoration={
      markings,
      mark=at position 0.5 with {\arrow{>}},
    },
    }
}

% From https://tex.stackexchange.com/a/136166/201701
\tikzset{
  invisible/.style={opacity=0},
  visible on/.style={alt={#1{}{invisible}}},
  alt/.code args={<#1>#2#3}{%
    \alt<#1>{\pgfkeysalso{#2}}{\pgfkeysalso{#3}} % \pgfkeysalso doesn't change the path
  },
}


{
\begin{frame}
\begin{tikzpicture}[
remember picture,
overlay,
scale=0.27,
>={Stealth[scale=1]},
every node/.style = { fill, circle, inner sep=0pt, minimum size=4pt },
cand/.append style = { >={Stealth[scale=0.7]} },
every loop/.append style = { min distance = 15mm },
]
\tikzset{shift={(current page.south west)},yshift=9.2cm,xshift=4cm}
\node (a) at (3,6) {};
\node (b) at (2,11) {};
\node (c) at (5,12) {};
\node[rd] (d) at (6,1) {};
\node (e) at (8,9) {};
\node (f) at (10,0) {};
\node (g) at (16,8) {};
\node[gr] (h) at (17,14) {};
\node (i) at (13,12) {};
\draw[->] (a) -- (c);
\draw[->] (b) -- (c);
\draw[->] (c) -- (e);
\draw[->] (f) -- (d);
\draw[->] (e) -- (f);
\draw[->] (g) -- (h);
\draw[->] (i) -- (h);
\draw[cand] (g) -- (e);
\draw[cand] (g) -- (d);
\draw[cand] (g) -- (f);
\path (d) edge [loop left] (d);
\path (h) edge [loop above] (h);
\end{tikzpicture}

\begin{tikzpicture}
\tikzset{
remember picture,
overlay,
shift={(current page.south west)},yshift=3.3cm,xshift=6.8cm,
}
\begin{axis}[
  xlabel={},
  xmax = 100000,
  cycle list = { [samples of colormap=8] },
  font=\footnotesize,
  scale=0.6,
  xmode = log,
  xmin = 1,
  grid,
  colormap name = colormap/jet,
  height=0.7\textheight,
  width=0.9\textwidth,
  xtick=\empty,
  ytick=\empty,
  every axis plot post/.append style = {
    line join=round,
    line width=1pt,
  },
]
\pgfplotstableread{data/converging/treeweight.dat}{\data}
\addplot+[dashed] table [y={arrow-random}] {\data};
\addplot+[dashed] table [y={arrow-star}] {\data};
\addplot+[dashed] table [y={arrow-mst}] {\data};
\addplot table [y={ivy-random}] {\data};
\addplot table [y={dynamicStar-random}] {\data};
\addplot table [y={minWeight-random}] {\data};
\addplot table [y={localMinPairs-random}] {\data};
\end{axis}
\end{tikzpicture}

\begin{textblock*}{\paperwidth}[0,0](0cm,0.3cm)
        \begin{center}
                \usebeamercolor[fg]{title}
                \textbf{\huge \inserttitle}
        \end{center}
\end{textblock*}
\begin{textblock*}{\paperwidth}[0,0](-0.5cm,7.0cm)
\flushright
\itshape \insertauthor\\
Advised by Pankaj Khanchandani and András Papp
\end{textblock*}
\begin{textblock*}{\paperwidth}[0,1](0.2cm,9.4cm)
        \flushleft
        \tiny \itshape \insertinstitute
\end{textblock*}
\end{frame}
}

\pgfplotsset{
  every axis legend/.append style={
    at={(0.5,1.03)},
    anchor=south
  },
  legend cell align = left,
  legend image post style = {
    line width = 2pt,
  },
  width=0.9\textwidth,
  height=0.5\textwidth,
  legend columns = 3,
  every axis/.append style = {
    xmode = log,
    xmin = 1,
    grid,
    colormap name = colormap/jet,
    legend style = { font = \scriptsize },
    xlabel = Number of requests,
    no markers,
  },
  every axis plot post/.append style = {
    thick,
    x={x},
    line join=round,
  },
}

% Explain the problem

\begin{frame}{Distributed Mutual Exclusion}

Single shared resource in network of nodes wanting exclusive access to it

\vspace{5mm}
\begin{center}
\begin{tikzpicture}[
scale=0.4,
>={Stealth[scale=1]},
dot/.style = { fill, circle, inner sep=0pt, minimum size=6pt },
font=\scriptsize,
]
\node[dot,gr] (a) at (5,5) {};
\node[dot] (b) at (8,9) {};
\node[dot] (c) at (10,0) {};
\node[dot,rd] (d) at (16,8) {};
\draw (a) --node[visible on=<2->, sloped, above]{2} (b);
\draw (a) --node[visible on=<2->, sloped, above]{3} (c);
\draw (a) --node[visible on=<2->, sloped, above]{5} (d);
\draw (b) --node[visible on=<2->, sloped, above]{4} (c);
\draw (b) --node[visible on=<2->, sloped, above]{4} (d);
\draw (c) --node[visible on=<2->, sloped, above]{3} (d);
\node[gr, left=0pt of a] {\footnotesize has token};
\node[rd, right=0pt of d] {\footnotesize wants token};
\end{tikzpicture}
\end{center}

\end{frame}


\begin{frame}{Arrow}
\centering

\begin{tikzpicture}[arvy-expl]
\node[v] (1) at (0,5) {a};
\node[q,visible on=<-5>] (2) at (7,5) {b};
\node[r,visible on=<6->] (2) at (7,5) {b};
\path (2) edge [loop above,visible on=<2->] (2);

\node[rd, above=3pt of 2,visible on=<1>] {\footnotesize wants token};
\node[v] (3) at (3,4) {c};
\node[v] (4) at (2,1) {d};
\node[r,visible on=<-4>] (5) at (6,0) {e};
\node[v,visible on=<5->] (5) at (6,0) {e};
\node[gr, above=0 of 5,visible on=<1>] {\footnotesize has token};


\draw[e] (1) -- (4);

\draw[e,visible on=<1>] (2) -- (3);
\draw[re,visible on=<2>] (2) --node[below]{\footnotesize request}  (3);
\draw[e,visible on=<3->] (3) -- (2);

\draw[e,visible on=<-2>] (3) -- (4);
\draw[re,visible on=<3>] (3) -- (4);
\draw[e,visible on=<4->] (4) -- (3);

\draw[e,visible on=<-3>] (4) -- (5);
\draw[re,visible on=<4>] (4) -- (5);
\draw[e,visible on=<5->] (5) -- (4);

\path[visible on=<-4>] (5) edge [loop below] (4);
\draw[re, gr, visible on=<5>] (5) --node[right=4pt]{\footnotesize token} (2);
\end{tikzpicture}

\end{frame}

\begin{frame}{Ivy}
\centering


\begin{tikzpicture}[arvy-expl]
\node[v] (1) at (0,5) {a};
\node[q,visible on=<-5>] (2) at (7,5) {b};
\node[r,visible on=<6->] (2) at (7,5) {b};
\path (2) edge [loop above,visible on=<2->] (2);

\node[rd, above=3pt of 2,visible on=<1>] {\footnotesize wants token};
\node[v] (3) at (3,4) {c};
\node[v] (4) at (2,1) {d};
\node[r,visible on=<-4>] (5) at (6,0) {e};
\node[v,visible on=<5->] (5) at (6,0) {e};
\node[gr, above=0 of 5,visible on=<1>] {\footnotesize has token};


\draw[e] (1) -- (4);

\draw[e,visible on=<1>] (2) -- (3);
\draw[re,visible on=<2>] (2) --node[below]{\footnotesize request}  (3);
\draw[e,visible on=<3->] (3) -- (2);

\draw[e,visible on=<-2>] (3) -- (4);
\draw[re,visible on=<3>] (3) -- (4);
\draw[e,visible on=<4->] (4) -- (2);

\draw[e,visible on=<-3>] (4) -- (5);
\draw[re,visible on=<4>] (4) -- (5);
\draw[e,visible on=<5->] (5) -- (2);

\path[visible on=<-4>] (5) edge [loop below] (4);
\draw[re, gr, visible on=<5>] (5) to[out=50,in=-80] node[right=4pt]{\footnotesize token} (2);
\end{tikzpicture}

\end{frame}

\begin{frame}{General Arvy}
\centering
\begin{tikzpicture}[arvy-expl]
\node[v] (1) at (0,5) {a};
\node[q,visible on=<-8>] (2) at (7,5) {b};
\node[r,visible on=<9->] (2) at (7,5) {b};
\path (2) edge [loop above,visible on=<2->] (2);

\node[rd, above=3pt of 2,visible on=<1>] {\footnotesize wants token};
\node[v] (3) at (3,4) {c};
\node[v] (4) at (2,1) {d};
\node[r,visible on=<-7>] (5) at (6,0) {e};
\node[v,visible on=<8->] (5) at (6,0) {e};
\node[gr, above=0 of 5,visible on=<1>] {\footnotesize has token};

\draw[e] (1) -- (4);

\draw[e,visible on=<1>] (2) -- (3);
\draw[re,visible on=<2>] (2) --node[below]{\footnotesize request}  (3);
\draw[cand,visible on=<3>] (3) -- (2);
\node[black!70,visible on=<3>] (cand) at (4,5.5) {\footnotesize{parent candidate}};
\draw[black!70,visible on=<3>] (cand) -- (4.9,4.6);
\draw[e,visible on=<4->] (3) -- (2);

\draw[e,visible on=<-3>] (3) -- (4);
\draw[re,visible on=<4>] (3) -- (4);
\draw[cand,visible on=<5>] (4) -- (2);
\draw[cand,visible on=<5>] (4) -- (3);
\draw[e,visible on=<6->] (4) -- (3);

\draw[e,visible on=<-5>] (4) -- (5);
\draw[re,visible on=<6>] (4) -- (5);
\draw[cand,visible on=<7>] (5) -- (2);
\draw[cand,visible on=<7>] (5) -- (3);
\draw[cand,visible on=<7>] (5) -- (4);
\draw[e,visible on=<8->] (5) -- (3);

\path[visible on=<-6>] (5) edge [loop below] (4);
\draw[re, gr, visible on=<8>] (5) -- node[right=4pt]{\footnotesize token} (2);
\end{tikzpicture}

\end{frame}

\begin{frame}{Edge Distance Minimizer}
\centering
\begin{tikzpicture}[arvy-expl,scale=2,font=\footnotesize]
\node[q] (a) at (0,0) {a};
\node[v] (b) at (0,2) {b};
\node[v] (c) at (2,3) {c};
\node[v,visible on=<4->] (d) at (3,1) {d};
\draw[e] (b) -- (a);

\draw[re, visible on=<1>] (b) --node{request} (c);
\draw[cand, visible on=<2>] (c) --node[sloped, above]{3} (a);
\draw[cand, visible on=<2>] (c) --node[sloped, above]{5} (b);
\draw[e,visible on=<3->] (c) --node[visible on=<3>, sloped, above]{3} (a);

\draw[re, visible on=<4>] (c) -- (d);
\draw[cand, visible on=<5>] (d) --node[sloped, above]{3} (a);
\draw[cand, visible on=<5>] (d) --node[sloped, above]{4} (b);
\draw[cand, visible on=<5>] (d) --node[sloped, above]{2} (c);
\draw[e,visible on=<6->] (d) --node[visible on=<6>, sloped, above]{2} (c);

\end{tikzpicture}
\end{frame}

\begin{frame}{Local Pair Distance Minimizer}
\centering

\begin{tikzpicture}[
arvy-expl,
scale=2,
font=\footnotesize,
ampersand replacement=\&,
]
\node[q] (a) at (0,0) {a};
\node[v] (b) at (0,2) {b};
\node[v] (c) at (2,3) {c};

\draw[re, visible on=<1>] (b) --node{request} (c);
\draw[e] (b) --node[left,visible on={<2,5>}]{4} (a);
\draw[cand, visible on={<2,5>}] (c) --node[sloped, above]{5} (b);
\draw[cand, visible on=<2>] (c) --node[sloped, above]{6} (a);
\draw[e,visible on=<3->] (c) -- (b);

\node[v,visible on=<4->] (d) at (3,1) {d};
\draw[re, visible on=<4>] (c) -- (d);
\draw[cand, visible on=<5>] (d) --node[sloped, above]{2} (a);
\draw[cand, visible on=<5>] (d) --node[sloped, above]{3} (b);
\draw[cand, visible on=<5>] (d) --node[sloped, above]{2} (c);
\draw[e,visible on=<6->] (d) -- (b);

\matrix[visible on=<2>] (m) at (4,2.8) [matrix of nodes]{
Total tree pair distance \\
b as new parent: $18$ \\
a as new parent: $20$ \\
};
\matrix[visible on=<5>] (m) at (4,2.8) [matrix of nodes]{
Total tree pair distance \\
c as new parent: $38$ \\
b as new parent: $36$ \\
a as new parent: $37$ \\
};
\end{tikzpicture}
\end{frame}

\begin{frame}{Dynamic Star}
For known probability distributions, there is a best star 2-approximation of the optimum\footnote{\scriptsize{\fullcite{Peleg}}}
\begin{block}{Idea}
\begin{itemize}
\item Measure frequency of requests for each node
\item Choose node with best estimated performance as star center
\end{itemize}
\end{block}
\end{frame}

\newcommand{\nodes}{
\node (0) at (0.9824309197096475,0.21600824801462992) {0};
\node (1) at (0.51630147561620086,0.3576546448862291) {1};
\node (2) at (0.6196010239841514,0.18770539954142618) {2};
\node (3) at (0.24250681611552394,0.47081365649116447) {3};
\node (4) at (0.6394214211843541,0.28528957307777114) {4};
\node (5) at (0.4496280158440922,0.8089357592613995) {5};
\node (6) at (0.8279469953621574,0.7162195110010463) {6};
\node (7) at (0.7622011483120192,0.55511162694708265) {7};
\node (8) at (0.5523928330845394,0.4259817586877246) {8};
\node (9) at (0.30737214539965896,0.35642649418227856) {9};
}

\begin{frame}{Dynamic Star}
\centering
\begin{tikzpicture}[
font=\scriptsize,
scale=10,
every node/.style = { fill,circle,inner sep=0pt, minimum size=5pt },
]
\nodes
\draw[thick] (0) -- (8);
\draw[thick] (1) -- (8);
\draw[thick] (2) -- (8);
\draw[thick] (3) -- (8);
\draw[thick] (4) -- (8);
\draw[thick] (5) -- (8);
\draw[thick] (6) -- (8);
\draw[thick] (7) -- (8);
\draw[thick] (9) -- (8);
\end{tikzpicture}
\end{frame}

\begin{frame}{Dynamic Star}
\centering
\begin{tikzpicture}[
font=\scriptsize,
scale=10,
every node/.style = { fill,circle,inner sep=0pt, minimum size=5pt },
]
\nodes
\draw[thick] (0) -- (1);
\draw[thick] (8) -- (1);
\draw[thick] (2) -- (1);
\draw[thick] (3) -- (1);
\draw[thick] (4) -- (1);
\draw[thick] (5) -- (1);
\draw[thick] (6) -- (1);
\draw[thick] (7) -- (1);
\draw[thick] (9) -- (1);
\end{tikzpicture}
\end{frame}

\begin{frame}{Results}
\begin{itemize}
\item Graph weights: Euclidean distances between uniformly random points in a unit square
\item Requests: Random or adversarial
\item Performance: Total time needed to satisfy all requests
\end{itemize}
\end{frame}

\begin{frame}{Tree Behavior}

\begin{tikzpicture}[
  p1/.style = { visible on = {<1,5>} },
  p2/.style = { visible on = {<2,5>} },
  p3/.style = { visible on = {<3,5>} },
  p4/.style = { visible on = {<4,5>} },
]
\centering
\begin{axis}[
  ylabel = \evalEdges,
  xmax = 100000,
  cycle list = { [samples of colormap=8] },
  font=\footnotesize,
]
\pgfplotstableread{data/converging/treeweight.dat}{\data}
\addplot+[dashed, p1] table [y={arrow-random}] {\data};
\addlegendentry[p1] {Uniformly random tree}
\addplot+[dashed,p2] table [y={arrow-star}] {\data};
\addlegendentry[p2] {Best star tree}
\addplot+[dashed, p4] table [y={arrow-mst}] {\data};
\addlegendentry[p4] {Min. sp. tree}
\addplot+[p1] table [y={ivy-random}] {\data};
\addlegendentry[p1] {Ivy}
\addplot+[p2] table [y={dynamicStar-random}] {\data};
\addlegendentry[p2] {\hyperref[alg:dynstar]{Dynamic Star}}
\addplot+[p4] table [y={minWeight-random}] {\data};
\addlegendentry[p4] {\hyperref[alg:ecm]{Edge dist min.}}
\addplot+[p3] table [y={localMinPairs-random}] {\data};
\addlegendentry[p3] {\hyperref[alg:lpm]{Local Pair Dist. Min.}}
\end{axis}
\end{tikzpicture}

\end{frame}

\begin{frame}{Best Heuristic for Random Requests}

\begin{tikzpicture}
\centering
\begin{axis}[
  ylabel = \evalTime,
  xmax = 1000000,
  cycle list = { [samples of colormap=5] },
  legend columns = 2,
  ymin=0,
  ymax=8,
  font=\footnotesize,
]
\pgfplotstableread{data/algs/time.dat}{\data}
\addplot table [y={arrow-star}] {\data};
\addlegendentry {Best star Arrow}
\addplot table [y={ivy-random}] {\data};
\addlegendentry {Ivy}
\addplot table [y={localMinPairs-random}] {\data};
\addlegendentry {\hyperref[alg:lpm]{Local Pair Dist. Min.}}

\end{axis}
\end{tikzpicture}
\end{frame}

\begin{frame}{Best Heuristic for Adversarial Requests}

\begin{tikzpicture}
\begin{axis}[
  xmax = 1000000,
  cycle list = { [samples of colormap=5] },
  legend columns = 2,
  ylabel=\evalTime,
  font=\footnotesize,
  ymin=0,
  ymax=8,
]
\pgfplotstableread{data/adversary/time.dat}{\data}
\addplot table [y={arrow-star}] {\data};
\addlegendentry {Best star Arrow}
\addplot table [y={ivy-random}] {\data};
\addlegendentry {Ivy}
\addplot table [y={localMinPairs-random}] {\data};
\addlegendentry {\hyperref[alg:lpm]{Local Pair Dist. Min.}}
\addplot table [y={dynamicStar-random}] {\data};
\addlegendentry {\hyperref[alg:dynstar]{Dynamic Star}}

\end{axis}
\end{tikzpicture}

\end{frame}


\begin{frame}{Conclusion}

\begin{itemize}
\item Arrow with the best star is very good for uniformly random requests
\item The Dynamic Star heuristic outperforms all others for adversarial requests
\item The Local Pair Distance Minimizer heuristic has decent performance and interesting properties
\end{itemize}

\end{frame}

\begin{frame}{}
\centering
Thanks for your attention

\vspace{1cm}
Questions?
\end{frame}

\begin{frame}{Best Tree for Arrow}

\begin{tikzpicture}
\centering
\begin{axis}[
  ylabel = \evalTime,
  xmax = 100000,
  cycle list = { [samples of colormap=5] },
  legend columns = 2,
  ymin = 1,
  ymax = 3,
  font=\footnotesize,
]
\pgfplotstableread{data/trees/time.dat}{\data}
\addplot table [y={arrow-random}] {\data};
\addlegendentry {Random tree}
\addplot table [y={arrow-mst}] {\data};
\addlegendentry {Minimum spanning tree}
\addplot table [y={arrow-shortpairs}] {\data};
\addlegendentry {\hyperref[tree:ampd]{Approx. min. pair dist.} tree}
\addplot table [y={arrow-star}] {\data};
\addlegendentry {\hyperref[tree:star]{Best star} tree}
\addplot table [y={arrow-shortestpairs}] {\data};
\addlegendentry {\hyperref[tree:mpd]{Min. pair dist.} tree}
\end{axis}
\end{tikzpicture}
\end{frame}


\begin{frame}{Ivy in Small Cliques}
Average request completion time between Arrow and Ivy in small cliques:
\begin{center}
\npdecimalsign{.}
\nprounddigits{3}
\begin{tabular}{ r | n{1}{3} | n{1}{3} }
  Node count & Arrow & Ivy \\
  \hline
  3 & 1.334 & {\color{ForestGreen} 1.250} \\
  \hline
  4 & 1.499 & {\color{ForestGreen} 1.443} \\
  \hline
  5 & {\color{ForestGreen} 1.600} & 1.603 \\
  \hline
  6 & {\color{ForestGreen} 1.665} & 1.739 \\
  \hline
  7 & {\color{ForestGreen} 1.713} & 1.859 \\
  \hline
  8 & {\color{ForestGreen} 1.749} & 1.963 \\
  \hline
  $\vdots$ & {\color{ForestGreen} $\vdots$} & $\vdots$ \\
\end{tabular}
\end{center}

\nocite{*}
\end{frame}

\begin{frame}{Recursive Clique}
\centering
\begin{tikzpicture}[
  font=\footnotesize,
]
\begin{groupplot}[
  ylabel = \evalTime,
  xmax = 1000000,
  cycle list = { [samples of colormap=5] },
  small,
  group style = {
    group size = 2 by 1,
    xlabels at = edge bottom,
    ylabels at = edge left,
  },
  scale=0.8,
]
\pgfplotstableread{data/reclique/time.dat}{\data}
\nextgroupplot[legend to name={CommonLegend}]
\addplot table [y={reclique}] {\data};
\addlegendentry {\hyperref[alg:reclique]{Recursive Clique}}
\addplot table [y={arrow-star}] {\data};
\addlegendentry {Best star Arrow}
\addplot table [y={arrow-mst}] {\data};
\addlegendentry {Min. sp. tree Arrow}
\addplot table [y={localMinPairs-random}] {\data};
\addlegendentry {\hyperref[alg:lpm]{Local Pair Dist. Min.}}
\addplot table [y={ivy-random}] {\data};
\addlegendentry {Ivy}
\node (t) at (axis cs:1000000,2.4) {};
\node (b) at (axis cs:1000000,1.5) {};
\draw[thick,red,dashed] (axis cs:100000,2.4) rectangle (axis cs:1000000,1.5);

\nextgroupplot[xmin=100000,ymin=1.5,ymax=2.4]
\addplot table [y={reclique}] {\data};
\addplot table [y={arrow-star}] {\data};
\addplot table [y={arrow-mst}] {\data};
\addplot table [y={localMinPairs-random}] {\data};
\end{groupplot}
\path[text=black] (group c1r1.north east) -- node[above=5pt]{\ref{CommonLegend}} (group c2r1.north west);
\draw[red,thick,dashed] (t) -- (group c2r1.north west);
\draw[red,thick,dashed] (b) -- (group c2r1.south west);
\end{tikzpicture}

\end{frame}


\end{document}
