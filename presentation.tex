\documentclass[12pt,hyperref={pdfpagelabels=false},usenames,dvipsnames]{beamer}
\let\Tiny=\tiny

\usepackage{xspace}
\usepackage{xmpmulti}
\usepackage{array}
\usepackage[absolute,overlay]{textpos}
\usepackage{forloop}
\usepackage{pgfplots}
\usepackage{subcaption}
\usepackage{tikz}
\usepackage{numprint}
\usetikzlibrary{pgfplots.groupplots}
\usetikzlibrary{calc}
\usetikzlibrary{positioning}
\usetikzlibrary{arrows.meta}
\usetikzlibrary{decorations.markings}

\setbeamerfont{frametitle}{series=\bfseries}
\setbeamerfont{title}{series=\bfseries,size=\huge}
\setbeamerfont{author}{series=\bfseries}

\setbeamertemplate{navigation symbols}{}

\title{Arvy Heuristics for \\
Distributed Mutual Exclusion}
\author{Silvan Mosberger}
\institute{ETH Zurich -- Distributed Computing Group -- www.disco.ethz.ch}

\begin{document}

\newcommand{\evalTime}{Average request time}
\newcommand{\evalEdges}{Average tree edge distance}

\tikzset{
  rd/.style = { red!70!black },
  bl/.style = { blue!70!black },
  gr/.style = { green!50!black },
  cand/.style = { dashed, black!70!white, ->, >={Stealth[scale=1]} },
  arvy-expl/.style =
    { v/.style = { circle, draw } % normal graph vertices
    , r/.style = { v, gr } % root nodes
    , q/.style = { v, rd } % currently making request
    , e/.style = { draw=black, -> }
    , re/.style = { postaction={decorate}, >={Stealth[scale=1.5]}, dashed, bl }
    , weight/.style = { dotted, black!30!white }
    , every node/.style = { inner sep=0pt, minimum size=18pt },
    , >={Stealth[scale=2]}
    , scale = 1.0
    , auto,
    decoration={
      markings,
      mark=at position 0.5 with {\arrow{>}},
    },
    }
}

% From https://tex.stackexchange.com/a/136166/201701
\tikzset{
  invisible/.style={opacity=0},
  visible on/.style={alt={#1{}{invisible}}},
  alt/.code args={<#1>#2#3}{%
    \alt<#1>{\pgfkeysalso{#2}}{\pgfkeysalso{#3}} % \pgfkeysalso doesn't change the path
  },
}


{
\begin{frame}
\begin{tikzpicture}[
remember picture,
overlay,
scale=0.27,
>={Stealth[scale=1]},
every node/.style = { fill, circle, inner sep=0pt, minimum size=4pt },
cand/.append style = { >={Stealth[scale=0.7]} },
every loop/.append style = { min distance = 15mm },
]
\tikzset{shift={(current page.south west)},yshift=7cm,xshift=4cm}
\node (a) at (3,6) {};
\node (b) at (2,11) {};
\node (c) at (5,12) {};
\node[rd] (d) at (6,1) {};
\node (e) at (8,9) {};
\node (f) at (10,0) {};
\node (g) at (16,8) {};
\node[gr] (h) at (17,14) {};
\node (i) at (13,12) {};
\draw[->] (a) -- (c);
\draw[->] (b) -- (c);
\draw[->] (c) -- (e);
\draw[->] (f) -- (d);
\draw[->] (e) -- (f);
\draw[->] (g) -- (h);
\draw[->] (i) -- (h);
\draw[cand] (g) -- (e);
\draw[cand] (g) -- (d);
\draw[cand] (g) -- (f);
\path (d) edge [loop left] (d);
\path (h) edge [loop above] (h);
\end{tikzpicture}

\begin{tikzpicture}
\tikzset{
remember picture,
overlay,
shift={(current page.south west)},yshift=2.8cm,xshift=6.8cm,
}
\begin{axis}[
  xlabel={},
  xmax = 100000,
  cycle list = { [samples of colormap=8] },
  font=\footnotesize,
  scale=0.6,
  xmode = log,
  xmin = 1,
  grid,
  colormap name = colormap/jet,
  height=0.7\textheight,
  width=0.9\textwidth,
  xtick=\empty,
  ytick=\empty,
  every axis plot post/.append style = {
    line join=round,
    line width=1pt,
  },
]
\pgfplotstableread{data/converging/treeWeight.dat}{\data}
\addplot+[dashed] table [y={arrow-random}] {\data};
\addplot+[dashed] table [y={arrow-star}] {\data};
\addplot+[dashed] table [y={arrow-mst}] {\data};
\addplot table [y={ivy-random}] {\data};
\addplot table [y={dynamicStar-random}] {\data};
\addplot table [y={localMinPairs-random}] {\data};
\addplot table [y={minWeight-random}] {\data};
\end{axis}
\end{tikzpicture}

\begin{textblock*}{\paperwidth}[0,0](0cm,0.7cm)
        \begin{center}
                \usebeamercolor[fg]{title}
                \textbf{\huge \inserttitle}
        \end{center}
        \usebeamercolor[fg]{normal text}
\end{textblock*}
\begin{textblock*}{\paperwidth}[0,0](-0.5cm,7.4cm)
        \flushright
        \itshape \insertauthor
        \usebeamercolor[fg]{normal text}
\end{textblock*}
\begin{textblock*}{\paperwidth}[0,1](0.2cm,9.4cm)
        \flushleft
        \usebeamercolor[fg]{institute}
        \tiny \itshape \insertinstitute
        \usebeamercolor[fg]{normal text}
\end{textblock*}
\end{frame}
}

\pgfplotsset{
  every axis legend/.append style={
    at={(0.5,1.03)},
    anchor=south
  },
  legend cell align = left,
  legend image post style = {
    line width = 2pt,
  },
  width=0.9\textwidth,
  height=0.5\textwidth,
  legend columns = 3,
  every axis/.append style = {
    xmode = log,
    xmin = 1,
    grid,
    colormap name = colormap/jet,
    legend style = { font = \scriptsize },
    xlabel = Number of requests,
    no markers,
  },
  every axis plot post/.append style = {
    thick,
    x={x},
    line join=round,
  },
}

% Explain the problem

\begin{frame}{Distributed Mutual Exclusion}

Single shared resource in network of nodes wanting exclusive access to it

\vspace{5mm}
\begin{center}
\begin{tikzpicture}[
scale=0.4,
>={Stealth[scale=1]},
dot/.style = { fill, circle, inner sep=0pt, minimum size=6pt },
font=\scriptsize,
]
\node[dot,gr] (a) at (5,5) {};
\node[dot] (b) at (8,9) {};
\node[dot] (c) at (10,0) {};
\node[dot,rd] (d) at (16,8) {};
\draw (a) --node[sloped, above]{2} (b);
\draw (a) --node[sloped, above]{3} (c);
\draw (a) --node[sloped, above]{5} (d);
\draw (b) --node[sloped, above]{4} (c);
\draw (b) --node[sloped, above]{4} (d);
\draw (c) --node[sloped, above]{3} (d);
\node[gr, left=0pt of a] {\footnotesize has token};
\node[rd, right=0pt of d] {\footnotesize wants token};
\end{tikzpicture}
\end{center}

\end{frame}


\begin{frame}{Arrow}
\centering

\begin{tikzpicture}[arvy-expl]
\node[v] (1) at (0,5) {a};
\node[q,visible on=<-5>] (2) at (7,5) {b};
\node[r,visible on=<6->] (2) at (7,5) {b};
\path (2) edge [loop above,visible on=<2->] (2);

\node[rd, above=3pt of 2,visible on=<1>] {\footnotesize wants token};
\node[v] (3) at (3,4) {c};
\node[v] (4) at (2,1) {d};
\node[r,visible on=<-4>] (5) at (6,0) {e};
\node[v,visible on=<5->] (5) at (6,0) {e};
\node[gr, above=0 of 5,visible on=<1>] {\footnotesize has token};


\draw[e] (1) -- (4);

\draw[e,visible on=<1>] (2) -- (3);
\draw[re,visible on=<2>] (2) --node[below]{\footnotesize request}  (3);
\draw[e,visible on=<3->] (3) -- (2);

\draw[e,visible on=<-2>] (3) -- (4);
\draw[re,visible on=<3>] (3) -- (4);
\draw[e,visible on=<4->] (4) -- (3);

\draw[e,visible on=<-3>] (4) -- (5);
\draw[re,visible on=<4>] (4) -- (5);
\draw[e,visible on=<5->] (5) -- (4);

\path[visible on=<-4>] (5) edge [loop below] (4);
\draw[re, gr, visible on=<5>] (5) --node[right=4pt]{\footnotesize token} (2);
\end{tikzpicture}

\end{frame}


\begin{frame}{Arrow}
\begin{block}{Idea}
Maintain rooted spanning tree pointing to the token
\end{block}
\end{frame}


\begin{frame}{Ivy}
\centering


\begin{tikzpicture}[arvy-expl]
\node[v] (1) at (0,5) {a};
\node[q,visible on=<-5>] (2) at (7,5) {b};
\node[r,visible on=<6->] (2) at (7,5) {b};
\path (2) edge [loop above,visible on=<2->] (2);

\node[rd, above=3pt of 2,visible on=<1>] {\footnotesize wants token};
\node[v] (3) at (3,4) {c};
\node[v] (4) at (2,1) {d};
\node[r,visible on=<-4>] (5) at (6,0) {e};
\node[v,visible on=<5->] (5) at (6,0) {e};
\node[gr, above=0 of 5,visible on=<1>] {\footnotesize has token};


\draw[e] (1) -- (4);

\draw[e,visible on=<1>] (2) -- (3);
\draw[re,visible on=<2>] (2) --node[below]{\footnotesize request}  (3);
\draw[e,visible on=<3->] (3) -- (2);

\draw[e,visible on=<-2>] (3) -- (4);
\draw[re,visible on=<3>] (3) -- (4);
\draw[e,visible on=<4->] (4) -- (2);

\draw[e,visible on=<-3>] (4) -- (5);
\draw[re,visible on=<4>] (4) -- (5);
\draw[e,visible on=<5->] (5) -- (2);

\path[visible on=<-4>] (5) edge [loop below] (4);
\draw[re, gr, visible on=<5>] (5) to[out=50,in=-80] node[right=4pt]{\footnotesize token} (2);
\end{tikzpicture}

\end{frame}

\begin{frame}{Ivy}
\begin{block}{Idea}
Choose node that sent request as new parent
\end{block}
\end{frame}

\begin{frame}{General Arvy}

\begin{tikzpicture}[arvy-expl]
\node[v] (1) at (0,5) {a};
\node[q,visible on=<-8>] (2) at (7,5) {b};
\node[r,visible on=<9->] (2) at (7,5) {b};
\path (2) edge [loop above,visible on=<2->] (2);

\node[rd, above=3pt of 2,visible on=<1>] {\footnotesize wants token};
\node[v] (3) at (3,4) {c};
\node[v] (4) at (2,1) {d};
\node[r,visible on=<-7>] (5) at (6,0) {e};
\node[v,visible on=<8->] (5) at (6,0) {e};
\node[gr, above=0 of 5,visible on=<1>] {\footnotesize has token};

\draw[e] (1) -- (4);

\draw[e,visible on=<1>] (2) -- (3);
\draw[re,visible on=<2>] (2) --node[below]{\footnotesize request}  (3);
\draw[cand,visible on=<3>] (3) -- (2);
\node[black!70,visible on=<3>] (cand) at (4,5.5) {\footnotesize{parent candidate}};
\draw[black!70,visible on=<3>] (cand) -- (4.9,4.6);
\draw[e,visible on=<4->] (3) -- (2);

\draw[e,visible on=<-3>] (3) -- (4);
\draw[re,visible on=<4>] (3) -- (4);
\draw[cand,visible on=<5>] (4) -- (2);
\draw[cand,visible on=<5>] (4) -- (3);
\draw[e,visible on=<6->] (4) -- (3);

\draw[e,visible on=<-5>] (4) -- (5);
\draw[re,visible on=<6>] (4) -- (5);
\draw[cand,visible on=<7>] (5) -- (2);
\draw[cand,visible on=<7>] (5) -- (3);
\draw[cand,visible on=<7>] (5) -- (4);
\draw[e,visible on=<8->] (5) -- (3);

\path[visible on=<-6>] (5) edge [loop below] (4);
\draw[re, gr, visible on=<8>] (5) -- node[right=4pt]{\footnotesize token} (2);
\end{tikzpicture}

\end{frame}

\begin{frame}{General Arvy}
\begin{block}{Idea}
Allow \textit{any} previous nodes on the request path as new parents
\end{block}
\end{frame}

\begin{frame}{Edge Distance Minimizer}
\centering
\begin{tikzpicture}
[arvy-expl, bn/.style={circle,draw}
,root/.style={bn,thick}
,be/.style={dashed,draw=blue!70!black,arrows={-Stealth[scale=1.5]}}
,req/.style={bn,red!70!black}
,auto,scale=1.3,font=\footnotesize]
\node[q] (n1) at (0,0) {a};
\node[bn] (n2) at (1,2) {b};
\node[bn] (n3) at (3.5,2.5) {c};
\node[bn] (n5) at (4,0) {d};
\draw[e] (n2) -- (n1);
\draw[e] (n3) -- (n2);
\draw[be] (n3) to[out=-40,in=70] node[bl]{\footnotesize{request}} (n5);
\draw[cand,white] (n5) -- node[sloped,above]{5} (n1);
\draw[cand,white] (n5) -- node[sloped,above]{2} (n2);
\draw[cand,white] (n5) -- node[sloped,above]{3} (n3);
\end{tikzpicture}

\end{frame}

\begin{frame}{Edge Distance Minimizer}
\centering
\begin{tikzpicture}
[arvy-expl, bn/.style={circle,draw}
,root/.style={bn,thick}
,be/.style={dashed,draw=blue!70!black,arrows={-Stealth[scale=1.5]}}
,req/.style={bn,red!70!black}
,auto,scale=1.3,font=\footnotesize]
\node[q] (n1) at (0,0) {a};
\node[bn] (n2) at (1,2) {b};
\node[bn] (n3) at (3.5,2.5) {c};
\node[bn] (n5) at (4,0) {d};
\draw[e] (n2) -- (n1);
\draw[e] (n3) -- (n2);
\draw[be,white] (n3) to[out=-40,in=70] node[bl,white]{\footnotesize{request}} (n5);
\draw[cand] (n5) -- node[sloped,above]{5} (n1);
\draw[cand] (n5) -- node[sloped,above]{2} (n2);
\draw[cand] (n5) -- node[sloped,above]{3} (n3);
\end{tikzpicture}

\end{frame}

\begin{frame}{Edge Distance Minimizer}
\centering

\begin{tikzpicture}
[arvy-expl, bn/.style={circle,draw}
,root/.style={bn,thick}
,be/.style={dashed,draw=blue!70!black,arrows={-Stealth[scale=1.5]}}
,req/.style={bn,red!70!black}
,auto,scale=1.3,font=\footnotesize]
\node[q] (n1) at (0,0) {a};
\node[bn] (n2) at (1,2) {b};
\node[bn] (n3) at (3.5,2.5) {c};
\node[bn] (n5) at (4,0) {d};
\draw[e] (n2) -- (n1);
\draw[e] (n3) -- (n2);
\draw[be,white] (n3) to[out=-40,in=70] node[bl,white]{\footnotesize{request}} (n5);
\draw[weight] (n5) -- node[sloped,above]{5} (n1);
\draw[e] (n5) -- node[sloped,above]{2} (n2);
\draw[weight] (n5) -- node[sloped,above]{3} (n3);
\end{tikzpicture}

\end{frame}

\begin{frame}{Edge Distance Minimizer}
\begin{block}{Idea}
Choose node with minimum edge distance as the new parent
\end{block}
\end{frame}

\begin{frame}{Local Pair Distance Minimizer}
\centering
\begin{tikzpicture}[arvy-expl,scale=2,font=\footnotesize]
\node[q] (0) at (0,0) {a};
\node[v] (1) at (0,2) {b};
\node[v] (2) at (2,3) {c};
\node[v,white] (3) at (3,1) {d};
\draw[e] (1) --node[left]{4} (0);
\draw[cand,white] (2) --node[below=2pt]{6} (0);
\draw[cand,white] (2) --node[above]{5} (1);
\draw[re] (1) to[out=60,in=180] node[above, blue!70!black]{\footnotesize{request}} (2);
\end{tikzpicture}
\end{frame}

\begin{frame}{Local Pair Distance Minimizer}
\centering
\begin{tikzpicture}[arvy-expl,scale=2,font=\footnotesize]
\node[q] (0) at (0,0) {a};
\node[v] (1) at (0,2) {b};
\node[v] (2) at (2,3) {c};
\node[v,white] (3) at (3,1) {d};
\draw[e] (1) --node[left]{4} (0);
\draw[cand] (2) --node[below=2pt]{6} (0);
\draw[cand] (2) --node[above]{5} (1);
\draw[re,white] (1) to[out=60,in=180] node[left, blue!70!black]{\footnotesize{request}} (2);
\end{tikzpicture}
\end{frame}

\begin{frame}{Local Pair Distance Minimizer}
\centering
\begin{tikzpicture}[arvy-expl,scale=2,font=\footnotesize]
\node[q] (0) at (0,0) {a};
\node[v] (1) at (0,2) {b};
\node[v] (2) at (2,3) {c};
\node[v,white] (3) at (3,1) {d};
\draw[e] (1) --node[left]{4} (0);
\draw[weight] (2) --node[below=2pt]{6} (0);
\draw[e] (2) --node[above]{5} (1);
\draw[re,white] (1) to[out=60,in=180] node[left, blue!70!black]{\footnotesize{request}} (2);
\end{tikzpicture}
\end{frame}

\begin{frame}{Local Pair Distance Minimizer}
\centering
\begin{tikzpicture}[arvy-expl,scale=2,font=\footnotesize]
\node[q] (0) at (0,0) {a};
\node[v] (1) at (0,2) {b};
\node[v] (2) at (2,3) {c};
\node[v] (3) at (3,1) {d};
\draw[e] (1) --node[left]{4} (0);
\draw[weight] (2) --node[below=2pt]{6} (0);
\draw[e] (2) --node[above]{5} (1);
\draw[cand] (3) --node[below]{2} (0);
\draw[cand] (3) --node[above]{3} (1);
\draw[cand] (3) --node[right]{2} (2);
\draw[re] (0) to[out=110,in=-110] node[left, blue!70!black]{\footnotesize{request path}} (1);
\draw[re] (1) to[out=60,in=180] (2);
\draw[re] (2) to[out=-30,in=90] (3);
\end{tikzpicture}
\end{frame}

\begin{frame}{Local Pair Distance Minimizer}
\centering
\begin{tikzpicture}[arvy-expl,scale=2,font=\footnotesize]
\node[q] (0) at (0,0) {a};
\node[v] (1) at (0,2) {b};
\node[v] (2) at (2,3) {c};
\node[v] (3) at (3,1) {d};
\draw[e] (1) --node[left]{4} (0);
\draw[weight] (2) --node[below=2pt]{6} (0);
\draw[e] (2) --node[above]{5} (1);
\draw[weight] (0) --node[below]{2} (3);
\draw[e] (3) --node[above]{3} (1);
\draw[weight] (2) --node[right]{2} (3);
\draw[re] (0) to[out=110,in=-110] node[left, blue!70!black]{\footnotesize{request path}} (1);
\draw[re] (1) to[out=60,in=180] (2);
\draw[re] (2) to[out=-30,in=90] (3);
\end{tikzpicture}
\end{frame}

\begin{frame}{Dynamic Star}
\begin{block}{Idea}
\begin{itemize}
\item Measure request probabilities $p_v$ for each node $v$
\item Create a star with the best center node over time
\end{itemize}
\end{block}
\end{frame}

\begin{frame}{Results}
\begin{itemize}
\item Performance: Total time needed to satisfy all requests
\item Graph costs: Uniformly random nodes in a unit square
\item Request sequence: Uniformly random or adversarial
\end{itemize}
\end{frame}




\begin{frame}{Tree Convergence}


\begin{tikzpicture}
\centering
\begin{axis}[
  ylabel = \evalEdges,
  xmax = 100000,
  cycle list = { [samples of colormap=8] },
  font=\footnotesize
]
\pgfplotstableread{data/converging/treeWeight.dat}{\data}
\addplot+[dashed] table [y={arrow-random}] {\data};
\addlegendentry {Uniformly random tree}
\addplot+[dashed] table [y={arrow-star}] {\data};
\addlegendentry {Best star tree}
\addplot+[dashed] table [y={arrow-mst}] {\data};
\addlegendentry {Min. sp. tree}
\addplot table [y={ivy-random}] {\data};
\addlegendentry {Ivy}
\addplot table [y={dynamicStar-random}] {\data};
\addlegendentry {\hyperref[alg:dynstar]{Dynamic Star}}
\addplot table [y={localMinPairs-random}] {\data};
\addlegendentry {\hyperref[alg:lpm]{Local Pair Dist. Min.}}
\addplot table [y={minWeight-random}] {\data};
\addlegendentry {\hyperref[alg:ecm]{Edge cost min.}}
\end{axis}
\end{tikzpicture}

\end{frame}

\begin{frame}{Best Heuristic for Random Requests}

\begin{tikzpicture}
\centering
\begin{axis}[
  ylabel = \evalTime,
  xmax = 1000000,
  cycle list = { [samples of colormap=3] },
  legend columns = 2,
  font=\footnotesize,
]
\pgfplotstableread{data/algs/time.dat}{\data}
\addplot table [y={arrow-star}] {\data};
\addlegendentry {Best star Arrow}
\addplot table [y={ivy-random}] {\data};
\addlegendentry {Ivy}
\addplot table [y={localMinPairs-random}] {\data};
\addlegendentry {\hyperref[alg:lpm]{Local Pair Dist. Min.}}

\end{axis}
\end{tikzpicture}
\end{frame}

\begin{frame}{Best Heuristic for Adversarial Requests}

\begin{tikzpicture}
\begin{axis}[
  xmax = 1000000,
  cycle list = { [samples of colormap=5] },
  legend columns = 2,
  ylabel=\evalTime,
  font=\footnotesize,
]
\pgfplotstableread{data/adversary/time.dat}{\data}
\addplot table [y={arrow-star}] {\data};
\addlegendentry {Best star Arrow}
\addplot table [y={ivy-random}] {\data};
\addlegendentry {Ivy}
\addplot table [y={localMinPairs-random}] {\data};
\addlegendentry {\hyperref[alg:lpm]{Local Pair Dist. Min.}}
\addplot table [y={dynamicStar-random}] {\data};
\addlegendentry {\hyperref[alg:dynstar]{Dynamic Star}}

\end{axis}
\end{tikzpicture}

\end{frame}


\begin{frame}{Conclusion}

\begin{itemize}
\item Arrow with a star is very good for uniformly random requests
\item The Dynamic Star heuristic outperforms all others for adversarial requests
\item The Local Pair Distance Minimizer heuristic is worth looking into for less node contention
\end{itemize}

\end{frame}

\begin{frame}{}
\centering
Thanks for your attention

\vspace{2cm}
Questions?
\end{frame}

\begin{frame}{Best Tree for Arrow}

\begin{tikzpicture}
\centering
\begin{axis}[
  ylabel = \evalTime,
  xmax = 100000,
  cycle list = { [samples of colormap=5] },
  legend columns = 2,
  ymin = 1,
  ymax = 3,
  font=\footnotesize,
]
\pgfplotstableread{data/trees/time.dat}{\data}
\addplot table [y={arrow-random}] {\data};
\addlegendentry {Random tree}
\addplot table [y={arrow-mst}] {\data};
\addlegendentry {Minimum spanning tree}
\addplot table [y={arrow-shortpairs}] {\data};
\addlegendentry {\hyperref[tree:ampd]{Approx. min. pair dist.} tree}
\addplot table [y={arrow-star}] {\data};
\addlegendentry {\hyperref[tree:star]{Best star} tree}
\addplot table [y={arrow-shortestpairs}] {\data};
\addlegendentry {\hyperref[tree:mpd]{Min. pair dist.} tree}
\end{axis}
\end{tikzpicture}
\end{frame}


\begin{frame}{Ivy in Small Cliques}
Average request completion time between Arrow and Ivy in small cliques:
\begin{center}
\npdecimalsign{.}
\nprounddigits{3}
\begin{tabular}{ r | n{1}{3} | n{1}{3} }
  Node count & Arrow & Ivy \\
  \hline
  3 & 1.334 & {\color{ForestGreen} 1.250} \\
  \hline
  4 & 1.499 & {\color{ForestGreen} 1.443} \\
  \hline
  5 & {\color{ForestGreen} 1.600} & 1.603 \\
  \hline
  6 & {\color{ForestGreen} 1.665} & 1.739 \\
  \hline
  7 & {\color{ForestGreen} 1.713} & 1.859 \\
  \hline
  8 & {\color{ForestGreen} 1.749} & 1.963 \\
  \hline
  $\vdots$ & {\color{ForestGreen} $\vdots$} & $\vdots$ \\
\end{tabular}
\end{center}

\end{frame}

\end{document}
